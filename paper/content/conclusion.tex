We have shown that dependently typed languages satisfy the core requirements from the \emph{Language Workbench Challenge}~\cite{DBLP:conf/sle/ErdwegSVBBCGHKLKMPPSSSVVVWW13}.

\Velo{}'s \emph{notation} as a \ac{dsl} is, by design, textual, and the internal core bounded by \Idris{}'s own notation requirements.
%
More importantly the \emph{semantics} (statics and dynamics) of \Velo{}
are verified as part of the implementation thanks to the dependently typed setting.
%
The weakest supported core criteria, unfortunately, is that for \emph{editor} support.
%
Languages created through \Idris{} do not get an editor, they are free form languages.
\Velo{}, and other languages, require that their parsers and elaborators be hand written and not derived.
This can change with future investigation.
\Idris{} has support for elaborator reflection
% ~\cite{DBLP:conf/icfp/ChristiansenB16} -- removed due to restrictions on bibliography.
which may provide a vehicle through which deriving parsers and elaborators can happen.


There are, however, more criteria from the language workbench feature model to consider:
%
semantic \& syntactic services for editors;
%
testing \& debugging;
%
and
%
composability.
%

With the rise of the \ac{lsp} it would be a good idea to look at how we can derive \ac{lsp} compatible language servers generically, thus addressing the missing provision of the optional semantic and syntactic services.
\Idris{} itself provides an \emph{IDE-Protocol}
% ~\cite{MANUAL:dtp/Christiansen2014} -- removed due to space limitations on bibliography
, and there is support for the \ac{lsp} in \Idris{} itself
\footnote{\url{https://github.com/idris-community/idris2-lsp}}.

Our languages also do not come with the ability to test and debug their
implementation.
%
If some of the features we have presented are fully formalised (e.g.\ execution),
others are only known to be scope-and-type preserving (e.g.\ \ac{cse}).
%
Therefore the dependently typed setting does not mean we do not need testing
anymore.
%
Prior work on generators for inductive families~\cite{DBLP:journals/pacmpl/LampropoulosPP18}
should allow us to bring property-based testing~\cite{DBLP:conf/icfp/ClaessenH00} to our core passes.


Finally there is language composability.
%
We have already seen (\Cref{sec:design:constants}) how \Velo{} can embed one language (that of primitives \& application) into the \ac{stlc} albeit for one language.
%
It would be an advantageous idea to support the reuse of existing languages, and their type-systems when designing new ones.
%
This is a hard problem:
%
One has to not only combine their semantics but also the remainder of the workbench tooling.
%
The \emph{Effects} library~\cite{DBLP:conf/icfp/Brady13} provides clues as to how we can compose \acp{dsl} horizontally but not vertically.


We strenuously believe that With future engineering satisfying these missing criteria is possible.

\jfdm{If there is space, a paragraph on Idris2 as a workbench would be good.}

%%% Local Variables:
%%% mode: latex
%%% TeX-master: "../paper"
%%% End:
