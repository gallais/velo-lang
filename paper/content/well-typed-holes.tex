\subsection{Well-Typed Holes}
\label{sec:design:holes}

Typed-holes are a useful feature in type-driven programming~\cite{DBLP:journals/pacmpl/OmarVCH19}, and supports the \emph{dropping in} of a special term, the hole, into a program such that during elaboration the type of that term will be synthesised.
User's can inspect the type of a hole and will be presented with the local context that will, hopefully, show useful terms to help fill the hole.

Within \Idris{} holes are lifted during elaboration to top-level definitions, allowing them to be placed in a position suitable for the elaborator to synthesis the hole's type.
To avoid such term manipulation in \Velo{}'s core we have taken a different approach, and our elaboration step seeks to resolve bound holes \emph{in situ}.

As with typing contexts we also provide a hole context to keep track of declared holes and \emph{when} they are declared.

\begin{Verbatim}
data Term : (ctxt : List Meta) -> (ctxt : SnocList Ty) -> (type : Ty) -> Type where
\end{Verbatim}

Here \IdrisType{Meta} is a simple data structure to snapshot:
the holes real name;
current state of the typing context;
and
the type of the hole itself.
This information is populated during the elaboration of \IdrisType{Raw} terms to \IdrisType{Holey} ones.
During the second elaboration step (\IdrisType{Holey} terms to \IdrisType{Terms}), we capture, and reason about, invariants about all the holes dug and the contexts in which they were dug.
Reasoning on these invariants supports coalescing of hole information to ensure that the reported context for the same-named holes that were dug\ldots

\jfdm{not sure how best to phrase this.}


%%% Local Variables:
%%% mode: latex
%%% TeX-master: "../paper"
%%% End:
