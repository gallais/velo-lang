\subsection{Well-Typed Holes}
\label{sec:design:holes}

Holes are a special kind of placeholder that programmers can use for parts of the program they have not yet written.
%
In a typed language, each hole will be assigned a type based on the context it is used in.

Special language support (the ability to inspect, refine, compute with, or fill an existing hole with an adequately typed term) enables \emph{type-driven programming}~\cite{DBLP:journals/pacmpl/OmarVCH19},
a practice by which the user enters in a dialogue with the compiler in order
to interactively build the program.
%
Barebones language support for type-driven programming should at least include the ability to:
%
(1) inspect the type of a hole and the local context it appears in;
%
(2) instantiate a hole with an adequately typed term;
%
and as well
%
(3) safely evaluate programs that still contain holes.
%
\Velo{} provides all three.
%
Our treatment of evaluation and instantiation are fairly standard, but our elaboration process is more interesting.

\Idris{}'s elaborator lifts holes to top-level declarations with no associated definition as it encounters them.
%
Because of this design choice users cannot mention the same hole explicitly in different places to state their intention that these yet unwritten terms ought to be the same.
%
Users can refer to the hole's solution by its name, but that hole is still very much placed in one specific position and it is from that position that \Idris{} infers its context.

In \Velo{}, however, we allow holes to be mentioned arbitrarily many times in
arbitrarily different local contexts.
%
In the following example, the hole \texttt{?h} occurs in two distinct contexts: $\emptyset,a,x$ and $\emptyset,a,y$.

\begin{center}
  \holeexamplegraph{}
\end{center}

As a consequence, a term will only fit in that hole if it happens to live in the shared common prefix of these two contexts ($\emptyset,a$).
%
Indeed, references to $x$ will not make sense in $\emptyset,a,y$ and vice-versa for $y$.


Our elaborator proceeds in two steps.
%
First, a bottom-up pass records holes as they are found and, in nodes with multiple subterms, reconciles conflicting hole occurrences by computing the appropriate local context restrictions.
%
This process produces a list of holes together with a \IdrisType{Holey} term that contains invariants ensuring these collected holes do fit in the term.
%
Second, a top-down pass produces a core \IdrisType{Term} indexed by the list of \IdrisType{Meta} (a simple record type containing the hole's name, the context it lives in, and its type).
%
Hole occurences end up being assigned a thinning embedding the metavariable's actual context into the context it appears in.

Although these intermediate representations are \Velo{}-specific, the technique
itself is general and can be reused by anyone wanting to implement well-scoped holes.

%%% Local Variables:
%%% mode: latex
%%% TeX-master: "../paper"
%%% End:
