\subsection{Being Positively Negative}
\label{sec:idioms:posneg}

A workhorse of dependently typed programming is using decidable
predicates that capture, and produce, positive information.
We can represent the results of a decidable function using \texttt{Dec}:

\begin{Verbatim}
data Dec a = Yes a | No (a -> Void)
\end{Verbatim}

\noindent
Notice that in the negative position (the \texttt{No} constructor) we do not return positive information.
We provide a proof of falsity.
When reasoning about our programs such proof of falsity is good.
When interacting with our programs such proof of falsity is not good.
Consider the following type signatures that specify two decidable procedures.
The first to determine if one natural number is greater than another.

\begin{Verbatim}
isGT : (x,y : Nat) -> Dec (GT x y)
\end{Verbatim}

\noindent
The other, \IdrisFunction{any}, determines if any element in the list satisfies a supplied predicate.

\begin{Verbatim}
any : (f  : (x : type) -> Dec (p x)) -> (xs : List type) -> Dec (Any p xs)
\end{Verbatim}

For \IdrisFunction{isGT} we know that if the result is negative we can safely assume that \IdrisType{GT} \IdrisImplicit{x y} is false.
With \IdrisFunction{any} we do not know for sure why the decidable procedure failed.
Was it because the list was empty?
or,
was it because all elements of \IdrisBound{xs} did not satisfy \IdrisImplicit{p}!?
Furthermore, we cannot programmatically determine the reason why an element in \IdrisBound{xs} did not satisfy \IdrisImplicit{p}?

To address these issues. we introduce \IdrisType{DecInfo} a variant of \IdrisType{Dec} that carries positive information in the negative case, as well as proofs-of-falsity.

\chcomment[id=gallais]
          {Do we want to spend precious space talking about DecInfo
            rather than going straight to the nicest solution?}
\chcomment[id=jfdm]
          {But we do not use the nicest solution...}

\begin{Verbatim}
data DecInfo e p = Yes p | No e (p -> Void)
\end{Verbatim}

With this simple change we can now report \emph{why} a decision procedure failed, as well as proof that it failed.
Our examples can now become:

\begin{Verbatim}
isGT : (x,y : Nat) -> DecInfo (LTE x y) (GT x y)
\end{Verbatim}

\noindent
and

\begin{Verbatim}
any : (f  : (x : type) -> Dec (q x) (p x))
   -> (xs : List type)
         -> DecInfo (All (\textbackslash{}x => Pair (q x) (Not (p x))) xs) (Any p xs)
\end{Verbatim}

\IdrisType{DecInfo} does, however, carry a proof of false.
A more interesting line of work would be to adopt \emph{constructive negation} to ensure that proofs of false arise from positive sources only~\cite{msfp/Atkey22}.
That is, given:

\begin{Verbatim}
Predicate : Type
Predicate = (pos ** neg ** (pos -> neg -> Void))
\end{Verbatim}

\noindent
we can recreate \IdrisType{DecInfo} as:

\begin{Verbatim}
DecInfo : Predicate -> Type
DecInfo (pos ** neg ** no) = Either neg pos
\end{Verbatim}

\noindent
Being \emph{positively negative} stems back to producing good error messages when implementing elaborators~\cite{DBLP:journals/jfp/McBrideM04}.
We are investigating the efficacy of being \emph{positively negative} and how that impacts program design in dependently-typed languages.

%%% Local Variables:
%%% mode: latex
%%% TeX-master: "../paper"
%%% End:
