\Velo{} is a programming language that will reduce terms down to values.
%
Within the dependently typed setting we can use dependent types to reason about our language's \emph{theoretical} type-soundness proof \emph{practically} by providing the language's operational semantics.
%
We can realise \Velo{}'s dynamic semantics either through (definitional) interpreters~\cite{10.1145/3093333.3009866,Augustsson1999edt}, or by providing a more traditional syntactic proof of type-soundness~\cite{DBLP:journals/iandc/WrightF94} but mechanised~\cite[Part 2: Properties]{plfa22.08}.
%
Through consideration of the latter approach we get, for free, verification that \Velo{}'s execution is correct by reusing the correctness proof as the evaluator.
%
Or rather what we in fact get for free is our evaluator as it is computed from our formal proof.

Like existing approaches~\cite[Part 2: Properties]{plfa22.08}, we have provided several data structures that capture how top-level terms reduce:

\begin{Verbatim}
data Redux : (this,that : Term metas ctxt type) -> Type where
\end{Verbatim}

\noindent
and how primitive operations reduce (\Cref{sec:design:constants}):

\begin{Verbatim}
data Reduxes : \{tys : List Ty\} -> (these, those : All (Term metas ctxt) tys)
             -> Type where
\end{Verbatim}

\noindent
together with functions that compute reduction steps.
%
We differ, however, from standard approaches by genericsing our proofs of progress such that the boilerplate for computing the reflexive transitive closure when reducing terms is tidied away in a shareable module.
%
Our top-level progress definition is thus parameterised by reduction and value definitions:

\begin{Verbatim}
data Progress : (0 value : Pred a) -> (0 redux : Rel a) -> (tm : a) -> Type
  where
    Done : \{0 tm : a\} -> (val : value tm) -> Progress value redux tm

    Step : \{this, that : a\}
        -> (step : redux this that) -> Progress value redux this
\end{Verbatim}

\noindent
and the result of execution, which is similarly parameterised, is as follows:

\begin{Verbatim}
data Result : (0 value : Pred a) -> (0 redux : Rel a) -> (this : a) -> Type
  where R : (that : a) -> (val : value that)
         -> (steps : RTList redux this that) -> Result value redux this
\end{Verbatim}

The benefit of our approach is that language designers need only to provide details of what reductions are, and how to compute a single reduction, the rest comes for free.
%
With the result of evaluation we also get the list of reduction steps made that can, optionally, be printed to show a trace of execution.

%%% Local Variables:
%%% mode: latex
%%% TeX-master: "../paper"
%%% End:
