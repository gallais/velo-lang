The design behind \Velo{} is purposefully unsurprising:
%
it is \emph{just} the \ac{stlc} extended with booleans and their conjunction, and natural numbers and their addition.
%
To promote the idea of interactive editing \Velo{} also supports well-typed holes.
%
Below we show an example \Velo{} program, which contains a single hole used in multiple locations, and an extract from the \acs*{repl}
session that lists the current set of holes.

\begin{center}
  \begin{minipage}[t]{0.55\linewidth}
\begin{Verbatim}
   let b = false
in let double
         = (fun x : nat => (add x x))
in let x = (double ?hole)
in         (double ?hole)
\end{Verbatim}
\end{minipage}
\hfill
  \begin{minipage}[t]{0.35\linewidth}
    \begin{Verbatim}
Velo> :holes
b : Bool
double : Nat -> Nat
----------
?hole : Nat
\end{Verbatim}
\end{minipage}

\end{center}

The \emph{featherweight} language design of \Velo{} helps us showcase better \emph{how} we can use dependently-typed languages as language workbenches~\cite{DBLP:journals/toplas/IgarashiPW01}.
%
Regardless of language complexity, \Velo{} is nonetheless a \emph{complete} language with a standard compiler pipeline, and \acs*{repl}.
%
A \ac{dsl} captures the language's concrete syntax, and a parser turns \ac{dsl} instances into raw unchecked terms.
%
Bidirectional type checking keeps type annotations to a minimun in the concrete syntax, and helps to better elaborate raw un-typed terms into a set of well-typed \acp{ir}:
%
\IdrisType{Holey} to support well-scoped typed holes;
%
and
%
\IdrisType{Terms} that is the core representation that captures our language's abstract syntax.
%
We present interesting aspects of our \ac{ir} design in \Cref{sec:design}.
%
Further, elaboration performs standard syntax transformations that turns let-bindings into function application thus reducing the size of our core.
%
From the core representation we also provide well-scoped \ac{cse} using co-\DeBruijn{} indexing (\Cref{sec:compiler-pass}), and we provide a verified evaluator to reduce terms to values (\Cref{sec:semantics}).


\todo{Show an example high-level trace of the output?}
\jfdm{This would be useful for a full-length paper, but not here.}

%%% Local Variables:
%%% mode: latex
%%% TeX-master: "../paper"
%%% End:
