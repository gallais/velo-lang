\todo{comment on cost of thinning as inductive values?}
\jfdm{We should do this in the extended edition}

Now that our core language is well-scoped by construction, our compiler passes must also be shown to be scope-preserving.
%
This is not a new requirement, merely it makes concrete a constraint that used to be enforced informally.
%
More importantly we show, with our compiler passes, that model-to-same-model transformation of our \ac{edsl} is possible, and that the infrastructure required is not bespoke to \Velo{}.

The purpose of \ac{cse} is to identify subterms that appear multiple times in the syntax tree, and to abstract over them to avoid needless recomputations at runtime.
%
In the following example for instance, we would like to let-bind $t$ before the application node (denoted \$) so that $t$ may be shared by both subtrees.

\begin{center}
  \cseexamplegraph{}
\end{center}

One of the challenges of \ac{cse}, as exemplified above, is that the term of interest may be burried deep inside separate contexts.
%
In our intrinsically scoped representation, $t$ in scope
$\Gamma,\, \ty{x}{\sigma}$ is potentially not actually
syntactically equal to a copy living in $\Gamma,\, \ty{a}{\tau},\, \ty{b}{\nu}$.
%
Indeed a variable $v$ bound in $\Gamma$ will, for instance, be
represented by the \DeBruijn{} index ($1+v$) in $\Gamma,\, \ty{x}{\sigma}$
but by the index ($2+v$) in $\Gamma,\, \ty{a}{\tau},\, \ty{b}{\nu}$.

If only terms were indexed by their exact \emph{support}!\jfdm{Is the technical term \emph{support} sufficiently well-known to not explain?}
%
We would not care about additional yet irrelevant variables that happen to be in scope.
%
The principled solution here is to switch to a different representation when performing \ac{cse}.
%
The co-\DeBruijn{} representation~\cite{DBLP:journals/corr/abs-1807-04085} provides exactly this guarantee.

%\subsection{Co-\DeBruijn{} representation}

In the co-\DeBruijn{} representation, every term is precisely indexed by its exact support.
%
That is to say that every subterm explicitly throws away the bound variables that are not mentioned in it.
%
By the time we reach a variable node, a single bound variable remains in scope:
%
precisely the one being referred to.

If we think of thinnings as sequences of 0/1 bits, stating whether a variable is kept or dropped, and admitting that such sequences can be represented as a list of either $\bullet$ (1) or $\circ$ (0).
Consider, for example, the $S$ combinator ($\lambda g. \lambda f. \lambda x. g x (f x)$) which is represented as follows in co-\DeBruijn{} notation (diagram taken from~\cite{MANUAL:draft/Allais22}).

\begin{center}
  \codebruijnexamplegraph{}
\end{center}

The first three $\lambda$ abstractions only use $\bullet$ in their thinnings because all of $g$, $f$, and $x$ do appear in the body of the combinator.
%
The first application node then splits the context into two: the first subterm ($g x$) drops $f$ while the second ($f x$) gets rid of $g$.
%
Further application nodes select the one variable still in scope for each leaf subterm: $g$, $x$, $f$, and $x$.

Using a co-\DeBruijn{} representation, we can easily identify shared subterms:
%
they need to not be mentioning any of the most local variables and be syntactically equal.
%
Our pass succesfully transforms the program on the left-hand side to the one
on the right-hand side where the repeated expressions \texttt{(add m n)}
and \texttt{(add n m)} have been let-bound.

\begin{center}
  \begin{minipage}[t]{0.4\textwidth}
    \begin{Verbatim}
let m = zero in
let n = (inc zero)
in (add (add (add m n) (add n m))
        (add (add n m) (add m n)))
    \end{Verbatim}
  \end{minipage}\hfill\begin{minipage}[t]{0.4\textwidth}
    \begin{Verbatim}
let m = zero       in
let n = (inc zero) in
let p = (add n m)  in
let q = (add m n)
in (add (add q p) (add p q))
    \end{Verbatim}
  \end{minipage}
\end{center}

This pass relies on the ability to have a compact representation of thinnings
(as the co-\DeBruijn{} representation makes heavy use of them),
and additionnally the existence of a quick equality test for them.
%
This is not the case in the implementation we include in \Velo{} but
it is a solved problem~\cite{MANUAL:draft/Allais22}.
