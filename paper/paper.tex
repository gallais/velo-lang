\documentclass[%
draft,
a4paper,
UKenglish,
cleveref,
autoref,
thm-restate,
%anonymous,
pdfa
]{oasics-v2021}

\usepackage{acronym}
\usepackage{xspace}
\usepackage[commandnameprefix=always]{changes}

\pdfoutput=1
\hideOASIcs %uncomment to remove references to OASIcs series (logo, DOI, ...), e.g. when preparing a pre-final version to be uploaded to arXiv or another public repository

%\graphicspath{{./graphics/}}%helpful if your graphic files are in another directory

\bibliographystyle{plainurl}% the mandatory bibstyle

\title{Type Theory as a Language Workbench} %TODO Please add

\author
{Jan {de Muijnck-Hughes}}
{University of Glasgow, UK}
{Jan.deMuijnck-Hughes@glasgow.ac.uk}
{https://orcid.org/0000-0003-2185-8543}
{} % TODO

\author
{Guillaume Allais}
{University of St Andrews, UK}
{gxa1@st-andrews.ac.uk} % Check
{https://orcid.org/0000-0002-4091-657X} % ORCID
{} % TODO

\author
{Edwin Brady}
{University of St Andrews, UK}
{ecb10@st-andrews.ac.uk} % Check
{} % ORCID
{} % TODO

% TODO mandatory, please use full name; only 1 author per \author macro; first two parameters are mandatory, other parameters can be empty. Please provide at least the name of the affiliation and the country. The full address is optional. Use additional curly braces to indicate the correct name splitting when the last name consists of multiple name parts.

\authorrunning{J. {de Muijnck-Hughes} and Guillaume Allais and Edwin Brady}
% TODO mandatory. First: Use abbreviated first/middle names. Second (only in severe cases): Use first author plus 'et al.'

\Copyright{Jan de Muijnck-Hughes and Guillaume Allais and Edwin Brady}
% TODO mandatory, please use full first names. LIPIcs license is "CC-BY";  http://creativecommons.org/licenses/by/3.0/

\ccsdesc[100]{\textcolor{red}{Replace ccsdesc macro with valid one}}
% TODO mandatory: Please choose ACM 2012 classifications from https://dl.acm.org/ccs/ccs_flat.cfm

\keywords{Dummy keyword}
% TODO mandatory; please add comma-separated list of keywords

\category{}
% optional, e.g. invited paper

\relatedversion{}

% \supplement{}
% optional, e.g. related research data, source code, ... hosted on a repository like zenodo, figshare, GitHub, ...
% \supplementdetails[linktext={opt. text shown instead of the URL}, cite=DBLP:books/mk/GrayR93, subcategory={Description, Subcategory}, swhid={Software Heritage Identifier}]{General Classification (e.g. Software, Dataset, Model, ...)}{URL to related version}
% linktext, cite, and subcategory are optional

% \funding{(Optional) general funding statement \dots}
% optional, to capture a funding statement, which applies to all authors. Please enter author specific funding statements as fifth argument of the \author macro.

% \acknowledgements{I want to thank \dots}
% optional

%\nolinenumbers %uncomment to disable line numbering

%Editor-only macros:: begin (do not touch as author)%%%%%%%%%%%%%%%%%%%%%%%%%%%%%%%%%%
\EventEditors{John Q. Open and Joan R. Access}
\EventNoEds{2}
\EventLongTitle{42nd Conference on Very Important Topics (CVIT 2016)}
\EventShortTitle{CVIT 2016}
\EventAcronym{CVIT}
\EventYear{2016}
\EventDate{December 24--27, 2016}
\EventLocation{Little Whinging, United Kingdom}
\EventLogo{}
\SeriesVolume{42}
\ArticleNo{23}
%%%%%%%%%%%%%%%%%%%%%%%%%%%%%%%%%%%%%%%%%%%%%%%%%%%%%%

\acrodef{dsl}[DSL]{Domain Specific Language}
\acrodef{edsl}[EDSL]{Embedded Domain Specific Language}
\acrodef{stlc}[STLC]{Simply-Typed Lambda Calculus}

\definechangesauthor[color=orange,name={Jan}]{jfdm}
\definechangesauthor[color=green, name={Guillaume}]{gallais}
\definechangesauthor[color=red, name={Edwin}]{edwin}

\newcommand{\Velo}{V{\'e}lo\xspace}

\begin{document}

\maketitle

% 1. State the problem
% 2. Say why it is an interesting problem
% 3. Say what your solution achieves
% 4. Say what follows from your solution

\begin{abstract}
We are planning to showcase dependently typed techniques to implement \& manipulate IRs.

The key components we are planning to treat are:

\begin{itemize}
\item Efficient de Bruijn representations
\item Co de Bruijn for CSE
\item Evaluation as Progress (with computation rules in one separate function)
\item Well scoped holes
\item Linear (in number of cases) decidable equality
\item Compact constant folding
\item Positive evidence for negative statements
\item Whole pipeline (parser, elaborator, REPL, evaluator, compiler passes)
\end{itemize}

Our ongoing work is available at:

\url{https://github.com/jfdm/velo-lang}
\end{abstract}

\section{Introduction}
\label{sec:introduction}

A \emph{Language Workbench} offers language designers an expressive environment in which to design and implement their \Acp{dsl}.
\chcomment[id=jfdm]{Needs extending.}

Dependently-typed languages such as Idris~\cite{DBLP:conf/ecoop/Brady21} provides programmers with an expressive environment in which to reason not only about their software programs but more importantly about the \emph{programming languages} in which we write our programs.
There is a plethora of work attesting to the ability of dependent-types to reason about the programming languages we all use, and textbooks such as \emph{Programming Language Foundations in Agda}~\cite{plfa22.08} and \emph{Software Foundations}~\cite{Pierce:SF2} detail how others can learn to develop mechanised formal descriptions of their languages.
\chcomment[id=jfdm]{Needs extending.}

The formal design of a programming language is, however, just one aspect of he language's lifecycle.
Programming languages are also to be used.
Dependently-typed languages such as Idris also supports the efficient execution of our `reasoned about' software programmes and the programming languages described.
\chcomment[id=jfdm]{Needs extending.}

This paper presents \Velo{} a minimal functional language based on the \ac{stlc} combined with an extended form of \emph{Hutton's Razor} that show cases how we can use languages, such as Idris, as a language workbench.
\chcomment[id=jfdm]{Needs extending.}
% C-c C-a ?todo

\section{Introducing Velo}
\label{sec:velo}

\chcomment{Whole pipeline (parser, elaborator, REPL, evaluator, compiler passes)}


\section{Language Design}
\label{sec:design}

\chcomment{Efficient De Bruijn Representation}
\chcomment{Compact constant folding}
\chcomment{Well-Typed Holes}

\section{Optimisation \& Execution}
\label{sec:compiler}


\chcomment{Co de Bruijn for CSE}
\chcomment{Evaluation as Progress (with computation rules in one separate function)}

\section{Good Programming Idioms}
\label{sec:idioms}


\chcomment{Positive evidence for negative statements}
\chcomment{Linear (in number of cases) decidable equality}


\section{Conclusion}
\label{sec:conclusion}


\bibliography{paper}

\end{document}
