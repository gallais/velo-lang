\chcomment{Whole pipeline (parser, elaborator, REPL, evaluator, compiler passes)}

\jfdm{Do we need to provide the formal description of us the STLC \& our extensions sufficiently \emph{well known}?}

\Velo{} is the \ac{stlc} extended with booleans and natural numbers with their respective conjunction and addition operations as primitives, together with well-scoped-typed holes to support simple interactive language design.
Such a featherweight language helps us to better showcase how we can use dependently-typed languages as language workbenches.
\jfdm{Do we need to evidence (through citation) benefits of featherweight languages?}

\begin{verbatim}
   let a = zero
in let b = (inc a)
in let c = (add a b)

in let d = true
in let e = false
in let f = (and d e)

in let foo = (fun x : nat => x)
in (foo ?hole)

\end{verbatim}

We have presented \Velo{} as a complete language with a standard compilation pipeline,
A concrete syntax is provided as a \ac{dsl}, complete with a parser to construct raw unchecked language terms.
Type checking elaborates these raw terms into a mechanically verified set of intermediate representations in which holes and compiler optimisations are performed.
For the final core representation we provide a verifiable evaluator to reduce terms to values.

\jfdm{Would it be good to show an example high-level trace of the output?}

\jfdm{Perhaps we can use this section as an outline and forward reference the latter sections?}

%%% Local Variables:
%%% mode: latex
%%% TeX-master: "paper"
%%% End:
